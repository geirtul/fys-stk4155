\section{Theory}

\subsection{Franke's function}
% Usikker på om den trenger eget underkapittel, men har det sånn
% inntil videre

Franke's function is a weighted sum of four exponentials and has been 
widely used when testing various interpolation and fitting algorithms.
The function is given by 

\begin{align*}
	f(x,y) &= 
	\frac{3}{4}\exp{\left(-\frac{(9x-2)^2}{4} 
	- \frac{(9y-2)^2}{4}\right)}
	+\frac{3}{4}\exp{\left(-\frac{(9x+1)^2}{49}
	- \frac{(9y+1)}{10}\right)} \\
	&+\frac{1}{2}\exp{\left(-\frac{(9x-7)^2}{4} 
	- \frac{(9y-3)^2}{4}\right)} 
	-\frac{1}{5}\exp{\left(-(9x-4)^2 - (9y-7)^2\right) }.
\end{align*}. 
~\cite{Project1}

\subsection{Linear methods for regression}

A linear model assumes that the regression function is linear in the 
inputs \(X_1, \dots, X_p\), and can give a polynomial representation
using basic expansions such as \(X_2 = X_1^2, X_3 = X_1^3\). 

If we have an input vector as described above and want to predict 
a real-valued output \(Y\), the linear model reads as
\begin{equation}
	f(X) = \beta_0 + \sum\limits_{j=1}^p X_j\beta_j
\end{equation}
~\cite{IntroStatistics}

For all three regression methods that we will look at in this report, 
there is a set of training data from which we want to estimate the parameter 
\(\beta\). 

\subsubsection{Ordinary Least Squares method}
The Ordinary Least Squares (OLS) method pick the coefficients
\(\beta = (\beta_0, \beta_1, \dots, \beta_p)^T\) to minimize the 
residual sum of squares given by
\begin{align}
	RSS(\beta) &= \sum\limits_{i=1}^N (y_i - f(x_i))^2\\
				&= \sum\limits_{i=1}^N 
				(y_i - \beta_0 - \sum\limits_{j=1}^p X_{ij}\beta_j )^2\\
\end{align}

This can be rewritten with matrix notation 

\begin{equation}
	RSS(\beta) = (\hat{y}-\vec{X}\beta)^T(\hat{y} - \hat{X}\beta)
\end{equation}. ~\cite{IntroStatistics} 
As this is explained in detail in the book as well as in the 
lecture notes ~\cite{LectureNotes-FysStk}, we will not derive
it here.

As we want to minimize the residual sum of squares, we require

\begin{equation}
	\frac{\partial RSS(\hat{\beta})}{\partial \hat{\beta}} 
	= 0 
	= \hat{X}^T(\hat{y} - \hat{X}\hat{\beta})
\end{equation}

This can be rewritten as 

\begin{equation}
	\hat{X}^T\hat{y} = \hat{X}^T\hat{X}\hat{\beta} 
\end{equation}.

We now has to assume that the matrix \(\hat{X}^T\hat{X}\) 
is invertible to get the solution

\begin{equation}
	\hat{\beta} = (\hat{X}^T\hat{X})^{-1}\hat{X}^T\hat{y}
\end{equation}
~\cite{LectureNotes-FysStk}~\cite{IntroStatistics}

\subsubsection{Ridge regression}
The OLS will not have a solution if the design matrix \(\hat{X}\) is singular
or near singular, as it depends in \(\hat{X}^T\hat{X}\) being invetertible. 
~\cite{LectureNotes-FysStk}

Ridge regression is known as a shrinkage method. It shrinks the regression
coefficients \(\hat{\beta}\) by imposing a penalty on their size
~\cite{IntroStatistics}. We are in other words adding a diagonal component
to the matrix to invert. From OLS we therefore make the following change

\begin{equation}
	\hat{X}^T\hat{X} \rightarrow \hat{X}^T\hat{X} + \lambda \hat{I}
\end{equation}

where I is the identity matrix.~\cite{LectureNotes-FysStk}.
Thus we get 

\begin{equation}
	RSS(\lambda) = (\hat{y}-\hat{X}\hat{\beta)}^T(\hat{y} 
	- \hat{X}\hat{\beta})
	- \lambda \hat{\beta}^T\hat{\beta}
\end{equation}
and
\begin{equation}
	\hat{\beta}^{ridge} = (\hat{X}^T\hat{X} 
	+ \lambda \hat{I})^{-1}\hat{X}^T\hat{y} 
\end{equation}
\(\lambda > 0 \) is a complexity parameter that controls the amount
of shrinkage. The larger calue of \(\lambda\) the greater amount of 
shrinkage.
\cite{IntroStatistics}

This can also be written in a non-vector format which makes 
explicit the size constraint on the parameters. 
It also makes it easier to see the difference from Lasso regression:

\begin{equation}
	\hat{\beta}^{ridge} = \operatorname{argmin}_{beta} 
							\sum\limits_{i=1}^N \left(y_i - \beta_0
							-\sum\limits_{j=1}^p x_{ij}\beta_j\right)^2
\end{equation}
subject to
\begin{equation}
	\sum\limits_{j=1}^p \beta_j^2 \leq t
\end{equation}
There is a one to one correspondence between \(\lambda\) and \(t\).
\cite{IntroStatistics} 

\subsubsection{Lasso regression}
Like ridge, lasso is a shrinkage method. The main difference from 
ridge is that lasso has no closed form expression. ~\cite{IntroStatistics} 

Similarly to ridge, the lasso estimate is given by

\begin{equation}
	\hat{\beta}^{lasso} = \operatorname{argmin}_{beta} 
							\sum\limits_{i=1}^N \left(y_i - \beta_0
							-\sum\limits_{j=1}^p x_{ij}\beta_j\right)^2
\end{equation}
subject to
\begin{equation}
	\sum\limits_{j=1}^p |\beta_j| \leq t
\end{equation}.

As you can see, the ridge penalty \(\sum_1^p \beta_j^2\) is replaced 
by a lasso penalty given by \(\sum_1^p |\beta_j|\). 
This makes the solution nonlinear in the \(y_i\) and is the reason
that the lasso solution don't have a closed form expression.

Lasso does a kind of continous subset selection because of the nature of 
the constraint. The reason for that is that making \(t\) sufficiently small,
will cause some of the coefficients to go to zero. ~\cite{IntroStatistics} 

\subsection{Error estimates}
To evaluate the different regression methods we have two main
measures that can be used. In both equation below \(\widetilde{y_i}\) is 
the predicted value of the i-th sample and that \(y_i\) is the corresponding
true value. \cite{Project1} 
The first is the Mean Squared Error (MSE) which is given by 
\begin{equation}
	MSE(\hat{y}, \widetilde{\hat{y}}) = \frac{1}{n}\sum\limits_{i=0}^{n-1}
	(y_i, \widetilde{y_i})^2 
\end{equation}.
The other is the \(R^2\) score which is defined by
\begin{equation}
	R^2(\hat{y}, \tilde{\hat{y}}) = 1 - \frac{\sum_{i=0}^{n - 1} (y_i - \tilde{		y}	y_i)^2}{\sum_{i=0}^{n - 1} (y_i - \bar{y})^2},
\end{equation}

where we have defined the mean value  of $\hat{y}$ as
\begin{equation}
\bar{y} =  \frac{1}{n} \sum_{i=0}^{n - 1} y_i.
\end{equation}~\cite{Project1}

\subsection{Resampling methods}
A resampling method is a tool that involves 
repeatingly drawing samples from a training data set and refitting a model
of interest on each sample, in order to obtain additional information 
about the fitted model. 
The reason for using a resampling method, is to obtain information that would
not be available from fitting the model only once using the original training
sample. ~\cite{LectureNotes-FysStk}

\subsubsection{Bootstrap}
Bootstrap is a resampling method, and is widely used.
The bootstrap method is based on the fact that \(\hat{\Theta} = \hat{\Theta}\)
is a random variable, because it is a function of random variables. 
Therefore it has a pdf, \(p(\vec{t})\). The aim of the bootstrap is to 
estimate \(p(\vec{t})\) by the relative frequency of \(\hat{\Theta}\)
~\cite{LectureNotes-FysStk}

The bootstrap method works with four different steps: 
\begin{enumerate}
		\item Draw with replacement \(n\) numbers for the observed variables
			\(\hat{x} = (x_1, x_2,\cdots, x_n)\).
		\item Define a vector \(\hat{x}^*\) containing the values which were 
			drawn from \(\hat{x}\). 
		\item Using the vector \(\hat{x}^*\) compute \(\hat{\Theta}^*\)
			by evaluating \(\hat{\Theta}\) under the observations 
			\(\hat{x}^*\).
		\item Repeat the process k times. 
\end{enumerate} 
~\cite{LectureNotes-FysStk}

		







