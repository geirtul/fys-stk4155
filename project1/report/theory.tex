\section{Theory}

\subsection{Franke's function}
% Usikker på om den trenger eget underkapittel, men har det sånn
% inntil videre

Franke's function is a weighted sum of four exponentials and has been 
widely used when testing various interpolation and fitting algorithms.
The function is given by 

\begin{align*}
	f(x,y) &= 
	\frac{3}{4}\exp{\left(-\frac{(9x-2)^2}{4} 
	- \frac{(9y-2)^2}{4}\right)}
	+\frac{3}{4}\exp{\left(-\frac{(9x+1)^2}{49}
	- \frac{(9y+1)}{10}\right)} \\
	&+\frac{1}{2}\exp{\left(-\frac{(9x-7)^2}{4} 
	- \frac{(9y-3)^2}{4}\right)} 
	-\frac{1}{5}\exp{\left(-(9x-4)^2 - (9y-7)^2\right) }.
\end{align*}. 

~\cite{Project1}

\subsection{Linear methods for regression}

A linear model assumes that the regression function is linear in the 
inputs \(X_1, \dots, X_p\), and can give a polynomial representation
using basic expansions such as \(X_2 = X_1^2, X_3 = X_1^3\). 

If we have an input vector as described above and want to predict 
a real-valued output \(Y\), the linear model reads as
\begin{equation}
	f(X) = \beta_0 + \sum\limits_{j=1}^p X_j\beta_j
\end{equation}
~\cite{IntroStatistics}

For all three regression methods that we will look at in this report, 
there is a set of training data from which we want to estimate the parameter 
\(\beta\). 




