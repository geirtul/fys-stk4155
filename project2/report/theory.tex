\section{Theory}

%Consider putting in: general on ML 

%Consider adding theory from project1? 
%Until then: 
As the theory behond the three regression methods used as well as error 
estimates and the bootstrap method is covered in the previous 
project, we will not restate it here.

\subsection{The Ising model} 
The ising model is a simple binary value system where the variables
in the model can take only two values. For examle \(\pm 1\) or \(0\) and \(1\). 
~\cite{Project2} 

We will look at the physicist's approach, and call the variables for spin.
~\cite{Project2}

Given an ensamble of random spin configurations we can assign an energy to
each state, using the 1D Ising model with nearest-neighbor interactions: 

\begin{equation}
	E = -J\sum\limits_{j=1}^N S_jS_{j+1} 
\end{equation}
J is the nearest-neighbor spin interaction, and \(S_j \epsilon {\pm 1}\) is a 
spin variable. N is the chain length. 
~\cite{HighBias}~\cite{Project2} 

In one dimension, this model has no phase transitions at finite temperature.
~\cite{Project2} 

To get a spin model with pairwise interactions between every pair of variables,
we choose the following model class: 

\begin{equation}
	E_{model}[S^i] = -\sum\limits_{j=1}^N\sum\limits_{k=1}^N J_{j,k} S_j^iS_{k}^i
\end{equation}
~\cite{HighBias} 

In this equation \(i\) represents a particular spin configuration. ~\cite{Project2}

The goal with this model is to determine the interaction matrix \(J_{j,k}\). 
As the model is linear in \(\mathbf{J}\), it is possible to use
linear regression.  

The problem can be recast on the form

\begin{equation}
	E_{model}^i = \mathbf{x}^i \cdot \mathbf{J}  
\end{equation}

\subsection{Logistic regression and classification problems}
Differently to linear regression, classificaltion problems 
are concerned with outcomes taking the form of descrete variables. 
For a specific physical problem, we'd like to identify its state, say whether
it is an ordered of disordered system. ~\cite{LectureNotes-FysStk}

Configurations representing states below the critical temperature are calles 
ordered states, while those above the critical temperature are called 
disorderes states. ~\cite{Project2} 


\subsection{Accuracy score}

\subsection{Gradient Decent solver}

\subsection{Cost functions} 






