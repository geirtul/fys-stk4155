\section{Conclusion}
In this project we have explored classification and regression applied to
the well-known Ising model. We showed that the ising model can be well
represented by linear regression methods like ordinary least squares,
Ridge, and LASSO. Moving on the the two-dimensional Ising model, we show that
in classifying the states as either ordered or disordered, the logistic
regressor is mediocre at best. However, the logistic regressor in this project 
does not utilize a regularization term (rather a momentum parameter).
Implementing that may change the outcome somewhat, but overfitting of the data
does not seem to be the issue.

Classifcation with both single- and multilayer perceptron is show to be very
successful in labling the input states, even for states in the critical
temperature region. Moreover, the neural network needs far less data for
training than the logistic regressor for fitting.

As a binary classifier, the current implementation is fine, but it can be 
expanded to function well for classifying data with more than two classes,
for example by applying the softmax function in the output layer, using
one node for each class. 
Then again, with packages like AutoKeras around, it is more suitable as
a learning exercise than for actual production use.
