\section{Discussion}
Based on figures \ref{fig:regression-mehta} and 
\ref{fig:regression-mehta-article}, we see that the figure in article 
~\cite{HighBias} is similar to our own figures. 
Compared to the plots in the article our plots shows less noice, 
and some other small differences, but we see that the tendencies are the same.

Looking at the R2 scores from figure \ref{fig:regression-r2}, we see 
that the R2 score for the training and test set follow each other closely. 
Again, comparing with articles plot, shown in figure 
\ref{fig:regression-r2-article}, we see the two have a similar shape 
but that there are a bigger diggerence between the R2 score for training 
data and the test data in the article. 
In the articles figure, the difference seems to be biggest for Ridge and 
OLS. As for our plot, it seems like Ridge has a bit bigger difference 
between the R2 score for training and test data than for the two other 
methods.

To discuss the logistic regression model, 
we look at figure \ref{fig:logistic-eta}. For some $\eta$ the model 
quickly rises to an approximate highest value, but jumps back down to 
the equivalent of guessing from time
to time. When approaching 30 epochs and more, this behaviour 
seems to diminish somewhat, and overall the etas that produce 
the best results ($10^{-5} - 10^{-2}$) have most of their values
in the higher points.
Also table ~\ref{tab:logistic-critical} is connected 
to the logistic method model. We see that the accuracy is best
for \(\eta = 10^{-3}\). 



