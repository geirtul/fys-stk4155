\documentclass[12pt, notitlepage]{article}
%\usepackage[backend=biber]{biblatex}
\usepackage{amsmath}
\usepackage{listings}
\usepackage{graphicx}
\usepackage{caption}
%\usepackage{subcaption}
\usepackage{commath}
\usepackage{hyperref}
\usepackage{url}
\usepackage{xcolor}
\usepackage{textcomp}
\usepackage{dirtytalk}
\usepackage{listings}
\usepackage{wasysym}
\usepackage{float}
\usepackage{listings}
\usepackage[linesnumbered,lined,boxed,commentsnumbered]{algorithm2e}
\usepackage{subfig}

% Packages from derivations_fullproblem.tex
\usepackage[squaren]{SIunits}
\usepackage{a4wide}
\usepackage{array}
\usepackage{cancel}
\usepackage{amsmath}
\usepackage{amsfonts}
\usepackage{amssymb}
\usepackage{graphicx}
\usepackage{enumerate}
\usepackage{titling}

% Parameters for displaying code.
\lstset{language=python}
\lstset{basicstyle=\ttfamily\small}
\lstset{frame=single}
\lstset{keywordstyle=\color{red}\bfseries}
\lstset{commentstyle=\itshape\color{blue}}
\lstset{showspaces=false}
\lstset{showstringspaces=false}
\lstset{showtabs=false}
\lstset{breaklines}

% Define new commands
\newcommand{\expect}[1]{\langle #1 \rangle}
% Add bibliography
\begin{document}


\title{FYS-STK4155 - Project 2}
\author{Geir Tore Ulvik, Idun Kløvstad}
\begin{titlingpage}
\maketitle
\begin{abstract}
In this work four methods to sample or manipulate input data to learning
algorithms are explored: Random over-sampling, SMOTE, ADASYN, and balanced
weighting.
Several way to measure classifier performance are used, and the performance 
of logistic regression and a random forest classifier is evaluated based 
on how they perform on a binary classification problem. 
The data set chosen is payment data from an important bank in Taiwan,
where the data describes credit card holders, and the goal is predicting
whether a customer will default the next payment or not. 
The results presented show that logistic regression can classify the credit 
card data perfectly by applying a balanced weighting of the inputs. 
Random forests reached an accuracy of ~$95\%$ (cross-validation score). 
A possible explanation for the success of weighting the inputs compared 
to the other sampling methods is that some features in the data set may be 
far more crucial in determining the class than others, 
and weighting is the most efficient way of emphasizing these features through 
the learning process. Random forests do not gain similar improvement with
the same weighting, showing most improvement when using random over-sampling.
\end{abstract}
\end{titlingpage}
\section{Introduction}
This report focuses on regression analysis and resampling methods. 
Three different regression methods will be combined with the bootsrap method 
and compared.
The regression models that will be looked at are the Ordinary Least Squares
(OLS) method, Ridge regression and Lasso regression. 

The first thing we will study is how to fit polynomials of different order
to a spesific two-dimensional function called Franke's function. 
Thereafter the same methods will be used, trying to reproduce digital terrain
data.

The report is structured so that there is first given a bit of theory 
about the different methods, then we will present how we chose to implement 
those methods and why, before we show and discuss the results of using the 
different methods and finally makes a conclusion of what we think are 
each methods strong and weak sides. 
%TODO kanskje rydde opp litt i hva konklusjonen skal være? 

\section{Theory}
\subsection{Imbalanced Data in Classification}
A common challenge in classification is imbalanced data, in which a large
amount of the labeled data belongs to just one or a few of the classes.
For binary classification, if 90\% of the data belongs to one of the classes,
then the classifier is likely to end up placing every single
input in that class, as it will bring its accuracy to 90\%. Techincally, this
accuracy is correct, but it's not very useful since the decision isn't at all
affected by the features of the input. Accuracy alone isn't a good enough
measure of performance to reveal this.

Fortunately, since this is common, a number of methods have been developed
to combat the issue, some of which are described below.

\subsection{Resampling and Weighting}
In resampling there are essentially two main categories: Under-sampling
over-sampling. The difference between them is that over-sampling works with
somehow generating more samples of the minority class, while under-sampling
uses a reduced amount of samples from the majority class.
Weighting the samples is a differente approach in which the samples labeled
as the minority class are weighted higher than the others during training.

\subsubsection{Naive Random Over-sampling}
A very straightforward way to balance a dataset, is to choose random samples 
from the minority class, with replacement, until there is roughly equal
amounts of samples belonging to each class.

\subsubsection{SMOTE}
SMOTE - Synthetic Minority Over-sampling Technique, as the name suggests will
actually synthesize samples from the minority class in order to over-sample,
instead of sampling with replacement. This is done by taking each minority 
class sample and introducing synthetic examples along the line segments joining 
any/all of the k minority class nearest neighbors. Depending upon the amount of 
over-sampling required, neighbors from the k nearest neighbors are randomly
chosen. The result of synthesizing rather than choosing with replacement is
that the decision region is forced to become more general.
See \cite{smote-article} for a more detailed explanation of the methods
involved.

\subsubsection{ADASYN}
\subsubsection{Balanced Weighting}



%\section{Method}

\subsection{Producing data and recast problem}
To generate the training data, we used the code given in 
project2. 
The code a system with size \(L=40\) and \(1000\) different 
ising states and returns the Ising-energies. 

Then the problem was recasted as a linear regression model, using 
the regression methods from project1. The theory behind 
the Ising model and the recasting is described in ~\ref{seq:isingtheory}.

\subsection{Estimating the coupling constant of the one-dimensional Ising model using linear regression}








\section{Results}
Results from the logistic regression are included in figures 
\ref{fig:logistic-basic}, \ref{fig:logistic-confmat}, \ref{fig:logistic-cumul}, 
\ref{fig:logistic-roc}.


\section{Results}
\begin{figure}[H]
\begin{center}
    \subfloat[]{\includegraphics[width=0.4\textwidth, height=0.25\textheight]{figures/logistic_none_cumul}} 
    \subfloat[]{\includegraphics[width=0.4\textwidth, height=0.25\textheight]{figures/logistic_none_confmat}} \\
    \subfloat[]{\includegraphics[width=0.5\textwidth]{figures/logistic_none_roc}}
\end{center}
\caption[caption]{Performance analysis for basic logistic regression.}
\label{fig:logistic-basic}
\end{figure}
\begin{figure}[H]
\begin{center}
    \subfloat[Random oversampling]{\includegraphics[width=0.4\textwidth, height=0.25\textheight]{figures/logistic_random_oversampling_confmat}} 
    \subfloat[ADASYN]{\includegraphics[width=0.4\textwidth, height=0.25\textheight]{figures/logistic_adasyn_confmat}} \\
    \subfloat[SMOTE]{\includegraphics[width=0.4\textwidth, height=0.25\textheight]{figures/logistic_smote_confmat}} 
    \subfloat[Balanced weighting]{\includegraphics[width=0.4\textwidth, height=0.25\textheight]{figures/logistic_none_balanced_confmat}}
\end{center}
\caption[caption]{Confusion matrices for the logistic regression model using different resampling methods to balance the data set.}
\label{fig:logistic-confmat}
\end{figure}
\begin{figure}[H]
\begin{center}
    \subfloat[Random oversampling]{\includegraphics[width=0.4\textwidth, height=0.25\textheight]{figures/logistic_random_oversampling_cumul}} 
    \subfloat[ADASYN]{\includegraphics[width=0.4\textwidth, height=0.25\textheight]{figures/logistic_adasyn_cumul}} \\
    \subfloat[SMOTE]{\includegraphics[width=0.4\textwidth, height=0.25\textheight]{figures/logistic_smote_cumul}} 
    \subfloat[Balanced weighting]{\includegraphics[width=0.4\textwidth, height=0.25\textheight]{figures/logistic_none_balanced_cumul}}
\end{center}
\caption[caption]{Cumulative gain chart for the logistic regression model using different resampling methods to balance the data set.}
\label{fig:logistic-cumul}
\end{figure}
\begin{figure}[H]
\begin{center}
    \subfloat[Random oversampling]{\includegraphics[width=0.4\textwidth, height=0.25\textheight]{figures/logistic_random_oversampling_roc}} 
    \subfloat[ADASYN]{\includegraphics[width=0.4\textwidth, height=0.25\textheight]{figures/logistic_adasyn_roc}} \\
    \subfloat[SMOTE]{\includegraphics[width=0.4\textwidth, height=0.25\textheight]{figures/logistic_smote_roc}} 
    \subfloat[Balanced weighting]{\includegraphics[width=0.4\textwidth, height=0.25\textheight]{figures/logistic_none_balanced_roc}}
\end{center}
\caption[caption]{ROC curves for the logistic regression model using different resampling methods to balance the data set.}
\label{fig:logistic-roc}
\end{figure}

\section{Discussion}
Based on figures \ref{fig:regression-mehta} and 
\ref{fig:regression-mehta-article}, we see that the figure in article 
~\cite{HighBias} is similar to our own figures. 
Compared to the plots in the article our plots shows less noice, 
and some other small differences, but we see that the tendencies are the same.

Looking at the R2 scores from figure \ref{fig:regression-r2}, we see 
that the R2 score for the training and test set follow each other closely. 
Again, comparing with articles plot, shown in figure 
\ref{fig:regression-r2-article}, we see the two have a similar shape 
but that there are a bigger diggerence between the R2 score for training 
data and the test data in the article. 
In the articles figure, the difference seems to be biggest for Ridge and 
OLS. As for our plot, it seems like Ridge has a bit bigger difference 
between the R2 score for training and test data than for the two other 
methods. 



\section{Conclusion}
In this work four methods to sample or manipulate input data to learning
algorithms are explored: Random over-sampling, SMOTE, ADASYN, and balanced
weighting.
Using Cumulative Gain Charts, Confusion Matrices and
ROC curves, and cross-validation, the performance of a logistic regression 
algorithm, and a random forest classifier is evaluated based on how they 
perform on a binary classification problem. 
The data set chosen is payment data from an important bank in Taiwan,
where the data describes credit card holders, and the goal is predicting
whether a customer will default the next payment or not. This allowed for
comparisons with \cite{ComparisonData}. The results presented show that
logistic regression can classify the credit card data perfectly by applying
a balanced weighting of the inputs. Random forests reached an accuracy of
~$95\%$ (cross-validation score). A possible explanation for the success of
weighting the inputs compared to the other sampling methods is presented. 
Specifically, some features in the data set may be far more crucial in
determining the class than others, and weighting is the most efficient way of
emphasizing these features through the learning process.
This works only for the logistic model, however. For random forests, the
most improvement was found when using random over-sampling.

For future studies, an in-depth look at which features are the most important,
and how the different models evaluate this, could be very interesting, as it
may reveal some underlying effects of the sampling techniques.


\appendix
\bibliographystyle{unsrt}
\bibliography{bibliography}
\end{document}
